\section{Problem set 1}      

\subsection{Problem 1}
1. Given the following equation in a group 
\begin{align*}
  x^{-1}yxz^{2} = 1
.\end{align*}
solve for \( y \).

\begin{solution}

  Let the group that these three elements \( x,y,z \) belong to be \( G \).
 \begin{align*}
  x^{-1} y x z^{2} &= 1 \\
  x x^{-1} y x z^{2} &= x   \\
  1_{G} y x z^{2} &= x   \\
  1_{G} y x z^{2} z^{-2} &= x z^{-2}   \\
  1 y x z^{2} z^{-2} &= x z^{-2}   \\
  y x 1_{G} &= x z^{-2}   \\
  y x x^{-1} &= x z^{-2} x^{-1}   \\
  y x x^{-1} &= x z^{-2} x^{-1}   \\
  y 1_{G} &= x z^{-2} x^{-1}   \\
  y &= x z^{-2} x^{-1} 
.\end{align*}
\end{solution}

Note: ask if need to explicitly state all these algebraic manipulations.
Also clarify if need to be explicit with identity with group as subscript.

\subsection{Problem 2}
In any group \( G \), show that \( (g^{-1})^{-1}  = g\) for any \( g \in G \). Show that 
for any \( m,n \in \mathbb{Z} \) that \( g^{m} g^{n} = g^{m + n} \) and \( (g^{m})^{n} = g^{mn} \).

\begin{solution}

  Let \( m,n \in \mathbb{Z}  \) then,
 \begin{align*}
   g^{m}g^{n} &= (g^{m})(g^{n}) \\
    &= \underbrace{(gg\ldots g)}_{\text{\( m \) terms}} \underbrace{\left( gg\ldots g \right)}_{\text{\( n \) terms}} \\
    &= \underbrace{gg\ldots g \: gg\ldots g}_{\text{\( m+n \) terms}} \\
    &= g^{m+n}
 .\end{align*} 
 Hence, this is true for all \( m,n \in \mathbb{Z} \).

 Now again let \( m,n \in \mathbb{Z} \), then,
 \begin{align*}
   (g^{m})^{n} &= \underbrace{g^{m} \ldots g^{m}}_{\text{\( n \) terms}}\\
               &= \underbrace{\underbrace{\left( g\ldots g \right)}_{\text{\( m \) terms}} \ldots \underbrace{\left( g\ldots g \right)}_{\text{\( m \) terms}}}_{\text{\( n \) times}}  \\
               &=  \underbrace{g\ldots g}_{\text{\( nm \) terms}} \\
               &= g^{nm}\\
               &= g^{mn} \\
  .\end{align*}
\end{solution}

Now let \( g \in G \) then,
\begin{align*}
  g^{-1} g = 1
.\end{align*}
So, this is true for all \( g \).
Hence, \( (g^{-1})^{-1} = g\) for all \( g \in G \).

\subsection{Problem 3}
Prove disprove or salvage if possible the following statement.
Given subgroups \( J, H \le G \). The union \( H \cup J \) is a subgroup of \( G \).

\begin{solution}
  The union \( H \cup J \) is not necessarily a subgroup of \( G \).
  We give a counter example to disprove this statement.

  Consider the groups \( \mathbb{Z}/3 \) and \( \mathbb{Z}/4 \le \mathbb{Z}\) equipped with integer addition as the group binary operation. Now, \( 2 \in \mathbb{Z} \) and \( 3 \in 4 \). \( 2 \times 3 = 6 \notin \frac{\mathbb{Z}}{3} \cup \frac{\mathbb{Z}}{4}\).
  Therefore this statement is false. However, we can salvage this statement
  by considering the intersection \( H \cap J \) instead of the union. This is indeed a subgroup of \( G \).
  Following is the proof.

  \begin{proof}
   Since, \( 1_{G} \in H\) and \( 1_{G} \in J \). \( 1_{G} \in H \cap  J \). Therefore, the identity element is in \( H \cap J \).

  Now, let \( x, y \in H \cap J  \). This means \( x,y \in J \implies xy \in J \) by closure under multiplication
  and \( x,y \in H \implies xy \in H \) by closure under multiplication. Hence \( xy \in H \cap J \).
  This is true for all \(xy \in  H \cap J \). Hence \( H \cap J \) is closed under group multiplication.

  Remains to prove closure under group inverse. Let \( x \in H \cap J \). Then \( x \in H \implies x^{-1} \in H \) and \( x \in J \implies x^{-1} \in  J\) due to closure under group inverse
  of \( H \) and \( J \). Hence \( x^{-1} \in H \cap J \). Hence, this is true for all \( x \in H \cap J \).

  We have proven closure under group multiplication and group inverse and also existence of identity. Hence 
  by subgroup theorem \( H \cap J \le G\).
  \end{proof}
\end{solution}

\subsection{Quesiton 4}
Let \( G \) be a group and \( H \subseteq G\). Show that \( H \) is a subgroup iff it is
non empty and for every \( h,j \in H \) we have \( hj^{-1} \in H \). This gives an
alternate characterization for subgroups. (there is an analogue here for subspaces do you know it?).

\begin{solution}
 Lets prove the forwards implication i.e \( H \le G \implies H  \) is non empty and \( hj^{-1} \in H\).

 Since \( H \) is a subgroup we know it contains an identity element so it must be nonempty.  

 Now, let \( h,j \in H \). Since \( H \) is closed under inverses we know that \( j^{-1} \in H \).
 Also, \( H \) is closed under group multiplication. Therefore, \( h j^{-1} \in H \).

 Therefore, this is true for all \( h,j \in H  \). Hence for all \( h,j \in H \) we have \( hj^{-1} \in H \).

 Now, we prove the reverse implication i.e \( H  \) is non empty and for all \( h,j \in H \) \( hj^{-1} \in H \implies H \le G \).

 Since, \( H \) is non empty we know that there exists an element \( h \in H  \). We also know that \( h,j^{-1} \in H \) for all
 \( h,j \in  H\). Hence, \( h h^{-1} \in H \). Therefore \( 1_{G} \in H \). Therefore,
 \( H \) contains the identity element.
 Note: Ask about this kind of variable naming. Is this too confusing perhaps ?.

 Now, we show existence of inverse. Let \( h \in H \) and we know \( 1_{G} \in H \), \( 1_{G}h^{-1} \in H\). Hence \( h^{-1} \in H\).

 Finally we show closure under group multiplication.
 Now, let \( h, j \in H \). Therefore, by closure of inverse proven above \( j^{-1} \in H  \). Hence,
 \( h(j^{-1})^{-1} \in H \implies hj \in H\). Therefore for all \(h,j \in H  \) we have \( hj \in H \).
 Hence we have proven closure under group multiplication, group inverse and existence of identity.

 Therefore, by subgroup theorem we have \( H \le G \).

 Hence we have proven both implications and therefore the statement.

 The vector space analogue is that \( V  \subseteq W \) is a vector subspace of \(W  \) with scalar field \( F \) equipped with vector addition (\( + \)) and scalar multiplication \( (*) \).
 iff \( V \) is non empty and \( \forall \mathbf{x},\mathbf{y} \in V   \) and \( \lambda, \mu   \in F \) we have \(\lambda \mathbf{x} + \mu  \mathbf{y}  \in V\).
 Note: ask if you were allowed to assume subgroup theorem here since I have.
\end{solution}

\subsection{Question 5}
Let \( G \) be a group with group multiplication \( \mu : G \times  G \to  G\). We define a 
new group multiplication by \( \nu : G \times G \to G : \left( g,g' \right) \mapsto \mu \left( g', g \right)\). We let \( G_{op} \) be the set \( G \) equipped with this map.
Show that \( G^{op}  \) is a group. (It is called the opposite group to \( G \)). Remark:
when there are two group structures on a set, then a product expression like \( gg'  \) can mean two
different things depending on which multiplication you use. A simple remedy is to introduce
more complicated notation like \( g  *  g' \coloneqq  v(g, g' ), gg' \coloneqq  µ(g, g' ) \). Then the relation between the
two group structures is \( g * g'  = g' g \).

\begin{solution}
We show associatovity of the binary operation. 
Let \( g, h, k \in G_{op} \). Then
\begin{align*}
  \left( g * h \right) * k = (hg) * k
  = k(hg)
  = khg
  = \left(kh  \right)g
  = g * \left(kh  \right)
  = g * \left(h * k  \right)
.\end{align*}
Since, this is true for all \( g,h,k \in G_{op}  \) we have associativity.

Now, we show existence of identity. We know \( 1_{G} \in G_{op} \). Let \( g \in G_{op} \), then 
\( 1_{G} * g = g1_{G} = g \). Similarly \( g * 1_{G} = 1_{G}g = g \). This is true for all \( g \in G_{op} \) Therefore, 
\( 1_{G} \) is also the identity of \( G_{op} \) and \( G_{opp} \) contains an identity element.


Now, show closure under inverse. Let \( g \in G_{op}  \) then \( g \in  G    \) and \( g^{-1} \in G \) as \( G \) is closed under
inverse being a group. So \( g^{-1} \in G_{op} \) (Note: kinda confusing notation beacuse this already 
kinda implies it being an inverse do need something to denote its an inverse specificllay in G). 
Now we need to show that this is an inverse in \( G_{op} \) as well.
\begin{align*}
   g * g^{-1} = g^{-1} g = 1_{G}
.\end{align*}

Similarly
\begin{align*}
   g^{-1} * g = gg^{-1} = 1_{G}
.\end{align*}
Therefore, \( g^{-1} \) is an inverse of \( g \) in \( G_{op} \) as well. This is true for all \( g \in G \).
Hence, \( G_{op} \) is closed under inverse.

Therefore, we have proven associativity, closure under group multiplication and group inverse for \( G_{op} \). 
Hence, \( G_{op} \) is a group by definition.
\end{solution}

\subsection{Question 6}
Let \( GL_{n}\left( \mathbb{Z} \right) \) be the set of \( n \times n \) matrices \( M \) with integer entries such that \( M -1 \) exists and also
has integer entries. Show that \( GL_{n}\left( \mathbb{Z} \right) \) forms a group when endowed with matrix multiplication.

\begin{solution}
 We know \(GL_{n}\left( \mathbb{R} \right)  \) is a group and that \( GL_{n}\left( \mathbb{Z} \right) \) is a subset of it. Therefore, we need to show
 that it is a subgroup.
 First we show closure under multiplication. Let \( A,B \in GL_{n}\left( \mathbb{Z} \right) \). Then, consider \( AB \).
 Let \( \left( a \right)_{ij} \) denote the entry at the \( i \)th row and \( j \)th column of \( AB \). 
 We know that this entry is going to be the dot product of the \( i \) row vector in \( A \) and
 \( j \)th column vector in \( B \). And since both these vectors only have integer components.
 The dot product is going to be an integer. Hence \( \left( a \right)_{ij} \in \mathbb{Z} \). This
 is true for all entries \( a_{ij} \) where \( i \) is the row number and \( j \) is the column number of the entry.
 Hence, \( AB \in GL_{n}\left( \mathbb{Z} \right)\). This is true for all \( A,B \in GL_{n}\left( \mathbb{Z} \right) \). Hence \( GL_{n}\left( \mathbb{Z} \right) \)
 is closed under multiplication.

 Now we prove closure under inverse. Let \( A \in GL_{n}\left( \mathbb{Z} \right) \). Then we know that 
 Todo: not quite sure about inverse. Look up algorithm for finding inverse.

 Now we show existence of identity. Since \( I_{n} \) only consists of \( 1  \) and 0 entries it is in \( GL_{n}\left( \mathbb{Z} \right) \).
 
 Therefore we have proven existence of identity and closure under inverse and multiplication. 
 Hence, by subgroup theorem \( GL_{n}\left( \mathbb{Z} \right) \) is a subgroup of \( GL_{n} \left( \mathbb{R} \right) \).
\end{solution}

\subsection{Question 7}
7. In this question, we identify \( 1 \times  1 \) matrices with their unique entry so that \( GL_{n}\left( \mathbb{C} \right) \) gets
identified with \( \mathbb{C}^{*} \) , the non-zero elements in \( \mathbb{C} \). Let \( \mu  \) be the subset of roots of unity of \( \mathbb{C}^{*} \) .
(Recall that a root of unity is a complex number \( \zeta \) such that \( \zeta^{n} = 1 \) for some positive integer \( n \)).
Show that, \( \mu  \) is a subgroup of \( \mathbb{C} \) . Show that the subset \( \mu n \) of \( n \)-th (not necessarily primitive)
roots of unity is in turn a subgroup of \( \mu  \).

\begin{solution}
  We know 
  \[ \mu  = \{e^{\frac{2k \pi i}{n} }\mid k,n \in \mathbb{Z}\}\]

  We prove closure under group multiplication. 
  Let \( x,y \in \mu  \) the \( x = e^{\frac{2h\pi i}{a}} \) for some \( h,a \in \mathbb{Z}\).
  Similarly. \( y = e^{\frac{2j\pi i}{b}} \) for some \( j,b \in \mathbb{Z} \). Now,
  \begin{align*}
    xy &= e^{\frac{2h\pi i}{a} + \frac{2j\pi i}{b}} \\
    &= e^{\frac{2hb\pi i + 2aj\pi i}{ab}} \\
    &= e^{\frac{2\pi i \left( hb + aj \right)}{ab}} \\
    &= e^{\frac{2\pi\left( hb + aj \right) i }{ab}} \in \mu 
  \end{align*}
  as \( ab \in \mathbb{Z} \) and \(  hb + aj \in \mathbb{Z} \). Since this is true for all \( x,y \in \mu  \) it is
  closed under group multiplication.

  Now we show closure under inverse. Let \( x \in \mu  \). Then
  \( x = e^{\frac{2h \pi i}{a}} \) for some \( h, a \in \mathbb{Z} \). So,
  \begin{align*}
    x^{-1} &= (e^{\frac{2 h \pi i}{a}})^{-1} \\
    &= e^{-\frac{2h \pi i}{a}} \\
    &= e^{\frac{2(-h) \pi i}{a}} \in \mu 
  .\end{align*}
  Since this is true for all \( x \in \mu  \). \( \mu  \) is closed under group inverse.

  Identity is in \( \mu  \) as \( 1  \) is a root of unity.

  Since, we have closure under inverse, group multiplication and existence of identity. By subgroup theorem, \( \mu  \) is a subgroup.

  The proof for \( nth \) roots being a subgroup is almost identical  just instead of \( a, b   \)    we have \( n \).
\end{solution}

\subsection{Question 8}
16. Describe explicitly, the subgroup \( H \) of \( GL_{2} \left( \mathbb{C} \right) \) generated by the matrices
\[
  \begin{pmatrix} 0 & 1 \\
    -1 & 0 
    \end{pmatrix}   \text{and }\begin{pmatrix} \zeta & 0 \\
                  0 & \zeta^{-1}
  \end{pmatrix} 
\] 
where \( \zeta  \) is a primitive \( n \)-th root of unity. This is the binary dihedral group.

\begin{solution}
  
\end{solution}

