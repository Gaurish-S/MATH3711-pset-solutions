\section{Problem set 2}      

\subsection{Solution 1}

\begin{align*}
  \langle 4,6 \rangle &= \{4^{i_{1}}6^{i_{2}} \ldots 4^{i_{r}}6^{i_{r}} \mid i_{k} \in \mathbb{Z}, k \in \{1,\ldots,r\}   \}   \\
&= \{4^{p}4^{q} \mid p \in \mathbb{Z}, q \in \mathbb{Z}\} 
.\end{align*}
Above this is incorrect. This does not form a group with multiplication its integers with addtion.
The answer is \( \langle 2 \rangle \).

\subsection{Solution 2}
\begin{align*}
  \langle \left( 1,0 \right) \rangle &= \{\left( 1,0 \right)^{i_{1}}\left( 0,1 \right)^{j_{1}} \ldots \left( 1,0 \right)^{i_{r}}\left( 0,1 \right)^{j_{r}} \mid i_{k} \in \mathbb{Z}, k \in \{1,\ldots,r\} \}    \\
                                     &= \{\left( p,q \right) \mid p,q \in \mathbb{Z}\} \\
.\end{align*}
Because, the binary operation over this group is vector addition.

Note: Ask if we need to prove, set equality in such questions or of obvious enough.
\subsection{Solution 3}
% Write forms of both
Let \( x, y \in \langle S \rangle \). Then we know \( x = x_{1}^{i_{1}}\ldots x_{r}^{i_{r}} \) where \( x_{1}, \ldots, x_{k} \in S\) and \( i_{1},\ldots,i_{k} \in \mathbb{Z}\) and
similarly \( y = y_{1}^{j_{1}}\ldots y_{r}^{j_{r}} \) where \( y_{1}, \ldots, y_{k} \in S\) and \( j_{1},\ldots,j_{k} \in \mathbb{Z}\).

Now, any \( x_{i}, y_{j}\) in the product forms of \( x  \) and \( y \) above commute since they are in \( S \).
Therefore, their powers also commute and \( x_{i}^{a} y_{j}^{b} =  y_{j}^{b} x_{i}^{a}\) for all \( a,b  \in \mathbb{Z}\).

So,
\begin{align*}
  xy &=x_{1}^{i_{1}}\ldots x_{r}^{i_{r}} y_{1}^{j_{1}}\ldots y_{r}^{j_{r}} \\
  &= y_{1}^{j_{1}}\ldots y_{r}^{j_{r}} x_{1}^{i_{1}}\ldots x_{r}^{i_{r}}\\
  &= yx
.\end{align*}

Since this is true for all \( x,y \in \langle S \rangle \). \( \langle S \rangle \) is abelian.


% Show any element in the product of x commutes with any in y.
% So you can move all Y's guys to the front and proof is concluded.

\subsection{Solution 4}
\begin{align*}
  \sigma &= \left( 1 \: 3 \: 6 \right) \left( 2 \: 5 \right) \\
  &= \left( 1 \: 6 \right) \left( 1 \: 3 \right) \left( 2 \: 5 \right) \\
.\end{align*}

Since this is an odd number of transpositions, \( \sigma  \) is odd. 
Lets compute \( \sigma \Delta\) to verify.

\begin{align*}
  & \Delta \left(x_{1}, \ldots,x_{6}  \right)  \\
  = &\left(x_{1} - x_{2}\right) \left( x_{1} - x_{3} \right) \left( x_{2} - x_{3} \right) \left( x_{1} - x_{4} \right) \left( x_{2} - x_{4} \right) \left( x_{3} - x_{4} \right) \\
    & \left( x_{1} - x_{5} \right) \left( x_{2} - x_{5}   \right) \left( x_{3} - x_{5} \right) \left( x_{4} - x_{5} \right) \left(x_{1} - x_{6}  \right) \left( x_{2} - x_{6} \right) \\
    & \left( x_{3} - x_{6} \right) \left( x_{4} - x_{6} \right) \left( x_{5} - x_{6} \right)
.\end{align*}

\begin{align*}
  \sigma \Delta \left( x \right) &= \Delta \left( x_{\sigma(1)}, \ldots, x_{\sigma(6)} \right) \\
  &= \Delta \left( x_{3}, x_{5}, x_{6}, x_{4}, x_{2}, x_{1}\right) \\
  &= \left(x_{3} - x_{5}\right) \left( x_{3} - x_{6} \right) \left( x_{5} - x_{6} \right) \left( x_{6} - x_{4} \right) \left( x_{5} - x_{4} \right) \left( x_{1} - x_{4} \right) \\
    & \left( x_{3} - x_{2} \right) \left( x_{5} - x_{2}   \right) \left( x_{6} - x_{2} \right) \left( x_{4} - x_{2} \right) \left(x_{3} - x_{1}  \right) \left( x_{5} - x_{1} \right) \\
    & \left( x_{6} - x_{1} \right) \left( x_{4} - x_{1} \right) \left( x_{2} - x_{1} \right) \\
  &= - \Delta  \left( x \right)
.\end{align*}
Hence, \( \sigma \Delta = - \Delta  \). As we expected as a concequence of \( \sigma  \) being odd.

\subsection{Question 5}
Note: Dont get this how can difference products be composed when their output is in \( \mathbb{R} \).

\subsection{Question 6}
Note: Clarify notation not sure how to interpret \( f(x_{1},\ldots,x_{n}) \). My guess would be this
function outputs a polynomial, \( (x-x_1)\ldots(x-x_n) \).

\subsection{Question 7}
 We first show
\( G \) is a disjoint union of its right cosets. Now, we know that
for any \( H \le G \), the relation \( h \equiv g \iff  h \in gH   \) for all \( g,h \in G \).
Is an equivalence relation with equivalence classes being left cosets of \( H \).
Now, let \( G = G^{op} \), then for any \( H \le  G \) 
\( h \equiv g \iff h \in g * H \) for all \( gH \) is an equivalence relation. 
With equivalence classes being left cosets of \( G^{op} \). Hence, 
the disjoint union of all the left cosets of \( G^{op} \) gives \( G^{op} \). However, 
\( g* H = Hg \) for any \( g \in G \) and \( H \le G \). Therefore, any left coset
is a right coset of \( G \). Hence the disjoint union of all the right cosets of 
\( G \) gives \( G^{op} = G \).

Let
\( \iota: G \to G: g \mapsto g^{-1}   \) be the inverse map of \( G \). Now,
let \( H \le G \) and \( g \in  G \). We want to show \( \iota \left( Hg \right) =  g^{-1}H\).

Let \( x \in \iota \left( Hg \right) \). Hence, there is a \( y \in Hg \) such that
\( \iota \left( y \right) = x \). \( y = hg \) for some \( h \in H \). So,
\( x = \iota \left( y \right) = g^{-1} h \in g^{-1}H\).
Since, this is true for all \( x \in \iota  \left( Hg \right) \). 
We have, \( \iota \left( Hg  \right) \subseteq g^{-1}H \).

Now we prove the reverse containment relation.
Let \( x \in g^{-1}H \). Then there is a \( h \in H  \) such that, \( x = g^{-1} h \).
We know, \( h^{-1}g \in Hg \). So, \( \iota \left( h^{-1}g \right) = g^{-1}h \), Hence
\( g^{-1}h \in  \iota \left( Hg \right)\) and \( x = g^{-1}h \). So, 
\( x \in \iota \left( Hg \right) \). Since, this is true for all \( x \in g^{-1}H \), 
\( g^{-1} H \). Therefore we have shown both containment relations, \( g^{-1}H = \iota (Hg) \).

We define \( \mu : H \setminus G \to G \setminus H: Hg \to \iota \left( Hg \right) = g^{-1}H\). This
is in injection as for any input \( Hg \) since \( g^{-1}H = \iota \left( Hg \right) \) and \( \iota  \) is a bijection and so
it is going to map only \( Hg \) to all elements in \( g^{-1}H \). It is clearly a surjection, since
for any coset in the codomain \(gH\) we can find an input coset \( Hg^{-1} \) in the domain, where
\( \mu \left( Hg^{-1} \right) = gH \). Hence, \( \mu  \) is a bijection between the set of left and right cosets.
So, there is no left or right index of a group.

\subsection{Question 8}
I think this is identical to the assignment question.

\subsection{Question 9}
Note that, \( H \cap K \) is a subgroup (Note: can I assume this is a subgroup, we've proved it in problem sets but wondering if we can use results like these.)
 of both \( H \) and \( K \). So according to lagranges theorem \( | H \cap K |  \) divides both \( | H | = 3 \) and \( | K | = 5 \). Since, \( 3 \) and \( 5 \) are prime
 there is only one number that can divide both, which is \( 1 \). Hence, \( | H \cap K | = 1 \). We know that a group must contain an identity element.
 So \( 1 \in H \cap K \). So, \( H \cap K = {1} \).

 \subsection{Question 10}

