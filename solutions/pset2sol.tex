\section{Problem set 2}      

\subsection{Solution 1}

\begin{align*}
  \langle 4,6 \rangle &= \{4^{i_{1}}6^{i_{2}} \ldots 4^{i_{r}}6^{i_{r}} \mid i_{k} \in \mathbb{Z}, k \in \{1,\ldots,r\}   \}   \\
&= \{4^{p}4^{q} \mid p \in \mathbb{Z}, q \in \mathbb{Z}\} 
.\end{align*}
Due to commutativity of integer multiplication.

\subsection{Solution 2}
\begin{align*}
  \langle \left( 1,0 \right) \rangle &= \{\left( 1,0 \right)^{i_{1}}\left( 0,1 \right)^{j_{1}} \ldots \left( 1,0 \right)^{i_{r}}\left( 0,1 \right)^{j_{r}} \mid i_{k} \in \mathbb{Z}, k \in \{1,\ldots,r\} \}    \\
                                     &= \{\left( p,q \right) \mid p,q \in \mathbb{Z}\} \\
.\end{align*}
Because, the binary operation over this group is vector addition.

Note: Ask if we need to prove, set equality in such questions or of obvious enough.
\subsection{Solution 3}
% Write forms of both
Let \( x, y \in \langle S \rangle \). Then we know \( x = x_{1}^{i_{1}}\ldots x_{r}^{i_{r}} \) where \( x_{1}, \ldots, x_{k} \in S\) and \( i_{1},\ldots,i_{k} \in \mathbb{Z}\) and
similarly \( y = y_{1}^{j_{1}}\ldots y_{r}^{j_{r}} \) where \( y_{1}, \ldots, y_{k} \in S\) and \( j_{1},\ldots,j_{k} \in \mathbb{Z}\).

Now, any \( x_{i}, y_{j}\) in the product forms of \( x  \) and \( y \) above commute since they are in \( S \).
Therefore, their powers also commute and \( x_{i}^{a} y_{j}^{b} =  y_{j}^{b} x_{i}^{a}\) for all \( a,b  \in \mathbb{Z}\).

So,
\begin{align*}
  xy &=x_{1}^{i_{1}}\ldots x_{r}^{i_{r}} y_{1}^{j_{1}}\ldots y_{r}^{j_{r}} \\
  &= y_{1}^{j_{1}}\ldots y_{r}^{j_{r}} x_{1}^{i_{1}}\ldots x_{r}^{i_{r}}\\
  &= yx
.\end{align*}

Since this is true for all \( x,y \in \langle S \rangle \). \( \langle S \rangle \) is abelian.


% Show any element in the product of x commutes with any in y.
% So you can move all Y's guys to the front and proof is concluded.

\subsection{Solution 4}
\begin{align*}
  \sigma &= \left( 1 \: 3 \: 6 \right) \left( 2 \: 5 \right) \\
  &= \left( 1 \: 6 \right) \left( 1 \: 3 \right) \left( 2 \: 5 \right) \\
.\end{align*}

Since this is an odd number of transpositions, \( \sigma  \) is odd. 
Lets compute \( \sigma \Delta\) to verify.

\begin{align*}
  & \Delta \left(x_{1}, \ldots,x_{6}  \right)  \\
  = &\left(x_{1} - x_{2}\right) \left( x_{1} - x_{3} \right) \left( x_{2} - x_{3} \right) \left( x_{1} - x_{4} \right) \left( x_{2} - x_{4} \right) \left( x_{3} - x_{4} \right) \\
    & \left( x_{1} - x_{5} \right) \left( x_{2} - x_{5}   \right) \left( x_{3} - x_{5} \right) \left( x_{4} - x_{5} \right) \left(x_{1} - x_{6}  \right) \left( x_{2} - x_{6} \right) \\
    & \left( x_{3} - x_{6} \right) \left( x_{4} - x_{6} \right) \left( x_{5} - x_{6} \right)
.\end{align*}

\begin{align*}
  \sigma \Delta &= \Delta \left( x_{\sigma(1)}, \ldots, x_{\sigma(6)} \right) \\
  &= \Delta \left( x_{3}, x_{5}, x_{6}, x_{4}, x_{2}, x_{1}\right) \\
  &= \left(x_{3} - x_{5}\right) \left( x_{3} - x_{6} \right) \left( x_{5} - x_{6} \right) \left( x_{6} - x_{4} \right) \left( x_{5} - x_{4} \right) \left( x_{1} - x_{4} \right) \\
    & \left( x_{3} - x_{2} \right) \left( x_{5} - x_{2}   \right) \left( x_{6} - x_{2} \right) \left( x_{4} - x_{2} \right) \left(x_{3} - x_{1}  \right) \left( x_{5} - x_{1} \right) \\
    & \left( x_{6} - x_{1} \right) \left( x_{4} - x_{1} \right) \left( x_{2} - x_{1} \right) \\
  &= - \Delta 
.\end{align*}

\subsection{Question 5}
